\documentclass[11pt]{article}

\usepackage[margin=1in]{geometry}
\usepackage{setspace}
\onehalfspacing
\usepackage{graphicx}
\graphicspath{report_images/}
\usepackage{appendix}
\usepackage{listings}
\usepackage{float}
\usepackage{amsthm}
% The next three lines make the table and figure numbers also include section number
\usepackage{chngcntr}
\counterwithin{table}{section}
\counterwithin{figure}{section}
% Needed to make titling page without a page number
\usepackage{titling}

% DOCUMENT INFORMATION =================================================
\font\titleFont=cmr12 at 11pt
\title {{\titleFont ECEN 429: Introduction to Digital Systems Design Laboratory \\ North Carolina Agricultural and Technical State University \\ Department of Electrical and Computer Engineering}} % Declare Title
\author{\titleFont Reporter: Nikiyah Beulah \\ \titleFont Partner: Chris Cannon} % Declare authors
\date{\titleFont February 15, 2018}
% ======================================================================

\begin{document}

\begin{titlingpage}
\maketitle
\begin{center}
	Lab 4
\end{center}
\end{titlingpage}

\section{Introduction}
The object of this lab is to introduce students to the topic of memory and show how we might represent and handle memory in VHDL. This lab also reiterated important important concepts about components that will be utilized in this project. By the end of this lab, we will be able to implement a basic memory module in VHDL with the ability to select specific values from the memory address.

\section{Background, Design Solution, and Results}

\subsection{Problem 1 ROM Implementation}

\subsubsection{Background}
We were instructed to implement a ROM module with a 4-bit input and a 3-bit output. Because there are 4-inputs, which correspond with the available addresses in this memory. 4-bits corresponds with 2\textsuperscript{4}, or 16 possible values. Therefore, we were able to derive that there are 16 addresses in this memory module. Because the output is only 3 bits, we knew that our memory word size was 3 bits, meaning that each memory address held 3 bits of data.

\subsubsection{Design Solution}

\subsubsection{Results}

\subsection{Problem 2 ROM Multiplexer}

\subsubsection{Background}

\subsubsection{Design Solution}

\subsubsection{Results}

\section{Conclusion}

\pagebreak

\textbf{Appendices}

\begin{appendices}

\section{Problem 1 VHDL Code}

\begin{lstlisting}[language=VHDL]

\end{lstlisting}

\section{Problem 1 Constraints File}
\begin{figure}[H]
\begin{center}
	%\includegraphics[width=0.5\textwidth]{./report-images/Part2/P2Const.png}
	\caption{\label{fig:Part1ConstFile}Constraints file for Problem 1.}
\end{center}
\end{figure}

\section{Problem 2 VHDL Code}
\begin{lstlisting}[language=VHDL]

\end{lstlisting}

\section{Problem 2 Constraints File}
\begin{figure}[H]
\begin{center}
	%\includegraphics[width=0.5\textwidth]{./report-images/Part2/P2Const.png}
	\caption{\label{fig:Part2ConstFile}Constraints file for Problem 2.}
\end{center}
\end{figure}

\end{appendices}
\end{document}
