\documentclass[11pt]{article}

\usepackage[margin=1in]{geometry}
\usepackage{setspace}
\onehalfspacing
\usepackage{graphicx}
\graphicspath{report_images/}
\usepackage{appendix}
\usepackage{listings}
\usepackage{float}

% DOCUMENT INFORMATION =================================================
\title {ECEN 429: Introduction to Digital Systems Design Laboratory \\ North Carolina Agricultural and Technical State University \\ Department of Electrical and Computer Engineering} % Declare Title
\author{Reporter: Nikiyah Beaulah\\ \and Partner: Chris Cannon} % Declare authors
\date{February 8, 2018}
% ======================================================================

\begin{document}

\maketitle % Render Title, Author, and Date

\begin{center}
Lab	2
\end{center}

\pagebreak

\section{Introduction}

\section{Background, Design Solution, and Results}

\subsection{Problem 1 Seven Segment Display}

\subsubsection{Background}
This problem prompted us to utilize the seven-segment display on the Basys3 board. Our input will be a 3-bit number so that we can provide input from 0-9, which is the range that can be displayed by a single seven-segment display. Seven-segment displays include, fittingly, seven segments. Those segments are conventionally referred to as a-g in accordance with Figure ~\ref{fig:sevenSegMap}.

\begin{figure}[h]
\begin{center}
\includegraphics[width=0.2\textwidth]{report-images/img1.png}
\caption{Seven-Segment Display Map}
\label{fig:sevenSegMap}
\end{center}
\end{figure}

\subsubsection{Design Solution}
As discussed in the background for this problem, a seven-segment display can only display a single-digit 0-9. We have to use a 4-bit number in order to represent 9 (1001b) however, 10-15 (1010b-1111b) cannot be displayed using a single seven-segment display. In our code, this will be treated as an error condition where no data will be displayed on the seven-segment display. The 4-bit input will be x3 down to x0, with x3 as the most significant bit, and the output bits have been defined in the assignment. The input pin assignments are defined in Table ~\ref{tab:sevenSigInput} and the output pin assignments as well as their segment are defined in Table ~\ref{tab:sevenSigOutput}. The truth table for our seven segment decoder is shown in Table ~\ref{tab:sevenSegTruthTable}. During our design process, we remembered that the seven-segment display is "active-low", meaning that a '0' is treated as "ON" and a '1' is treated as "OFF". This was important when designing our truth table.

\begin{table}[h]
\begin{center}
	\begin{tabular}{| l | l | l |}
		\hline
		Bit & Label & Address \\ \hline
		x3 & Switch 3 & V17 \\ \hline
		x2 & Switch 2 & V16 \\ \hline
		x1 & Switch 1 & W16 \\ \hline
		x0 & Switch 0 & W15 \\ \hline
	\end{tabular}
	\caption{\label{tab:table-name}Input pin assignments for seven-segment decoder.}
	\label{tab:sevenSegInput}
\end{center}
\end{table}

\begin{table}[h]
\begin{center}
	\begin{tabular}{| l | l | l |}
		\hline
		Bit & Seven-Segment & Addres \\ \hline
		out1 & a & W7 \\ \hline
		out2 & b & W6 \\ \hline
		out3 & c & U8 \\ \hline
		out4 & d & V8 \\ \hline
		out5 & e & U5 \\ \hline
		out6 & f & V5 \\ \hline
		out7 & g & U7 \\ \hline
	\end{tabular}
	\caption{\label{tab:table-name}Output pin assignments for seven-segment decoder.}
	\label{tab:sevenSegOutput}
\end{center}
\end{table}

\begin{table}[h]
\begin{center}
	\begin{tabular}{| l | l | l | l | l | l | l | l | l | l | l |}
		\hline
		x3 & x2 & x1 & x0  & out1 & out2 & out3 & out4 & out5 & out6 & out7\\ \hline
		0 & 0 & 0 & 0 & 0 & 0 & 0 & 0 & 0 & 0 & 1 \\ \hline
		0 & 0 & 0 & 1 & 1 & 0 & 0 & 1 & 1 & 1 & 1 \\ \hline
		0 & 0 & 1 & 0 & 0 & 0 & 1 & 0 & 0 & 1 & 0 \\ \hline
		0 & 0 & 1 & 1 & 0 & 0 & 0 & 0 & 1 & 1 & 0 \\ \hline
		0 & 1 & 0 & 0 & 1 & 0 & 0 & 1 & 1 & 0 & 0 \\ \hline
		0 & 1 & 0 & 1 & 0 & 0 & 1 & 0 & 0 & 1 & 0 \\ \hline
		0 & 1 & 1 & 0 & 0 & 1 & 0 & 0 & 0 & 0 & 0 \\ \hline
		0 & 1 & 1 & 1 & 0 & 0 & 0 & 1 & 1 & 1 & 1 \\ \hline
		1 & 0 & 0 & 0 & 0 & 0 & 0 & 0 & 0 & 0 & 0 \\ \hline
		1 & 0 & 0 & 1 & 0 & 0 & 0 & 0 & 1 & 0 & 0 \\ \hline
		1 & 0 & 1 & 0 & 1 & 1 & 1 & 1 & 1 & 1 & 1 \\ \hline
		1 & 0 & 1 & 1 & 1 & 1 & 1 & 1 & 1 & 1 & 1 \\ \hline
		1 & 1 & 0 & 0 & 1 & 1 & 1 & 1 & 1 & 1 & 1 \\ \hline
		1 & 1 & 0 & 1 & 1 & 1 & 1 & 1 & 1 & 1 & 1 \\ \hline
		1 & 1 & 1 & 0 & 1 & 1 & 1 & 1 & 1 & 1 & 1 \\ \hline
		1 & 1 & 1 & 1 & 1 & 1 & 1 & 1 & 1 & 1 & 1 \\ \hline
	\end{tabular}
	\caption{\label{tab:table-name}Truth table for seven-segment decoder.}
	\label{tab:sevenSegTruthTable}
\end{center}
\end{table}

\subsection{Results}

\subsection{Problem 2 1:2 Decoder}

\subsubsection{Background}

\subsubsection{Design Solution}

\subsubsection{Results}

\subsection{Problem 3 SUM of a Full Adder}

\subsubsection{Background}

\subsubsection{Design Solution}

\subsubsection{Results}

\section{Conclusion}

\end{document}
