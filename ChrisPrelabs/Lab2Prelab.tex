\documentclass[11pt]{article}

\usepackage[margin=1in]{geometry}
\usepackage{setspace}
\onehalfspacing
\usepackage{graphicx}
\usepackage{listings}

% DOCUMENT INFORMATION =================================================
\title {ECEN 429: Introduction to Digital Systems Design Laboratory \\ North Carolina Agricultural and Technical State University \\ Department of Electrical and Computer Engineering} % Declare Title
\author{Chris Cannon} % Declare authors
\date{February 1, 2018}
% ======================================================================

\begin{document}
	
\maketitle

\begin{center}
	Lab 2 Prelab
\end{center}

\pagebreak

\section{Introduction}

The objective of this lab is to further master the use of the Vivado Design Suite in conjunction with the Basys3 Development Board. We will also practice implementing circuits of a given design on the Basys3 board. This lab consists of three parts, utilizing the seven segment displays for the first time. I anticipate that the seven segment display will offer a new technical challenge to complete.

\section{Background, Design Solution, and Results}

\subsection{Problem 1 Seven Segment Display}

Using out1-out7 as the outputs from my VHDL program, I can map these to pins a-g on a seven segment display. See Table 1.

\begin{table}[h]
\begin{center}
	\begin{tabular}{| l | l | l |}
		\hline	
		Output & Segment & Pin \\ \hline
		out1 & a & W7 \\ \hline
		out2 & b & W6 \\ \hline
		out3 & c & U8 \\ \hline
		out4 & d & V8 \\ \hline
		out5 & e & U5 \\ \hline
		out6 & f & V5 \\ \hline
		out7 & g & U7 \\ \hline
	\end{tabular}
	\caption{\label{tab:table-name}FPGA pin assignments for seven segment display.}
\end{center}	
\end{table}

\subsection{Problem 2 1:2 Decoder}

This decoder will convert a single bit into a 2-bit one hot format. The truth table for this simple circuit is Table 2. In order to perform this in VHDL, I would use the following simple program:

\begin{lstlisting}[language=VHDL]
library IEEE;
use IEEE.STD_LOGIC_1164.ALL;

-- This declares an entity that will have the 2 needed inputs and 2 outputs	
entity decoder is port (a, sel : in bit; out1, out2 : out bit);
end entity decoder;

architecture decoder_arch of decoder is
begin
	-- out1 is the result of 'A' AND the opposite of 'SEL'
	out1 <= a and (not sel);
	-- out2 is the result of 'A' AND 'SEL'
	out2 <= a and sel;
end architecture decoder_arch;	
\end{lstlisting}

\begin{table}[h]
\begin{center}
	\begin{tabular}{| l | l | l | l |}
		\hline
		a & sel & out1 & out2 \\ \hline
		0 & 0 & 0 & 0 \\ \hline 
		0 & 1 & 0 & 0 \\ \hline
		1 & 0 & 1 & 0 \\ \hline
		1 & 1 & 0 & 1 \\ \hline
	\end{tabular}
	\caption{\label{tab:table-name}Truth table for 1:2 decoder.}
\end{center}	
\end{table}

\subsection{Problem 3 SUM of a Full Adder}

I will derive the expression for SUM by first reference the truth table of the full-adder in Table 3. From table 3 I can derive that SUM will be equal to:

\begin{lstlisting}[language=VHDL]
((not a) and b and ci) or (a and (not b) and ci) or 
(a and b and (not ci)) or (a and b and ci)
\end{lstlisting}

\begin{table}[h]
\begin{center}
	\begin{tabular}{| l | l | l | l | l |}
		\hline
		A & B & CI & SUM & CO \\ \hline
		0 & 0 & 0 & 0 & 0 \\ \hline
		0 & 0 & 1 & 1 & 0 \\ \hline
		0 & 1 & 0 & 1 & 0 \\ \hline
		0 & 1 & 1 & 0 & 1 \\ \hline
		1 & 0 & 0 & 1 & 0 \\ \hline
		1 & 0 & 1 & 0 & 1 \\ \hline
		1 & 1 & 0 & 0 & 1 \\ \hline
		1 & 1 & 1 & 1 & 1 \\ \hline
	\end{tabular}
	\caption{\label{tab:table-name}Truth table for full adder.}
\end{center}	
\end{table}


\section{Conclusion}

After completion of the exercises, I feel well prepared for the lab. While the seven-segment display mapping was new, it was an easy challenge to overcome by simply employing the documentation. I think writing the truth table for the full adder when it wasn't strictly required will give me an additional resource that will be useful as I work through the lab.

\end{document}